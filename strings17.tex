\begin{marginfigure}%
  \includegraphics[width=\linewidth]{picture.png}

\end{marginfigure}
\begin{abstract}
\noindent
This is the string project   .
\end{abstract}

\normalsize

%this generates 1cm of vertical space
\vspace{1cm}
\section{ Common string operations}

".

\marginnote[30pt]{This is string Source code mean .}

\begin{shaded}
\begin{verbatim}
The string module contains a number of useful constants
and classes,
 as well as some deprecated legacy functions that are also
 available as methods on strings.
 In addition, Pythons built-in string classes support 
 the sequence type methods described in the Sequence Types 
  str,
 unicode, list, tuple, bytearray, buffer, xrange section,
 and also the string-specific methods described in the String
  Methods section.
 To output formatted strings use template strings or 
 the % operator described in the String Formatting Operations
  section.
 Also, see the re module for string functions based on regular
  expressions
\end{verbatim}
\end{shaded}

\vspace{1cm}

\section{Representing Python Code in Your Assignment}

\marginnote[40pt]{Here is some code can show you how to use the str in python.}

\begin{framed}
\begin{verbatim}
while b**e<c:
     x=a%(b**(e+1)
     y=x/(b**(e))
     ans=str(y)+ans
     a=a-x
     e=e+1
print ans

\end{verbatim}
\end{framed}

\marginnote[40pt]

\begin{shaded}
\begin{verbatim}
3111
\end{verbatim}
\end{shaded}
