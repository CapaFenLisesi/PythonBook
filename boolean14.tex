\section{Boolean Operations — and, or, not}



\includegraphics[scale=0.5]{boolean.jpg}
\caption{The Universe}
\label{fig:univerise}
\end{figure}









\includegraphics[scale=0.5]{000.jpg}
\caption{The Universe}
\label{fig:univerise}
\end{figure}






  
1.OR

This is a short-circuit operator, so it only evaluates the second argument if the first one is False.


2.And

This is a short-circuit operator, so it only evaluates the second argument if the first one is True.


3.Not


not has a lower priority than non-Boolean operators, so not a == b is interpreted as not (a == b), and a == not b is a syntax error








\section{Comparisons}
Comparison operations are supported by all objects. They all have the same priority (which is higher than that of the Boolean operations). Comparisons can be chained arbitrarily; for example, x < y <= z is equivalent to x < y and y <= z, except that y is evaluated only once (but in both cases z is not evaluated at all when x < y is found to be false).

This table summarizes the comparison operations
\includegraphics[scale=0.5]{123.jpg}
\caption{The Universe}
\label{fig:univerise}
\end{figure}


a)equal"=="

use == check the the date is equal or not. example 3 == 7 is false.



b)strictly greater than ">"



c)greater than or equal">="




d)strictly less than"<"




e)less than or equal"<="


