\chapter{Time Library}

The time library is a library that comes default with Python. The
time library uses the computer's clock in order to provide various
commands and functions that revolve around time. In time-based programming,
time starts at a point called the epoch. In the case of most Unix-based
operating systems, the epoch is 00:00 1 January 1970.


\section{time.time()}

Returns the time in seconds as a floating-point number, since the
epoch.

\begin{verbatim}
>>> import time
>>> time.time()
1447441291.782
>>> #taken at 14:01:40 13 November 2015
\end{verbatim}


